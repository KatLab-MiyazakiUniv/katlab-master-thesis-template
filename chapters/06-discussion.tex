\chapter{考察}\label{cha:Evaluation}

本章では、論文執筆における章立てとセクション構成について説明する。

\section{章の構成}

論文は複数の章で構成される。各章は \verb|\chapter{}| コマンドで定義する。

\subsection{章番号の自動付与}

LaTeX では章番号が自動的に付与される。
例えば、本章は第\ref{cha:Evaluation}章である。

\subsection{章のラベル付け}

各章には \verb|\label{}| を付けることで、他の箇所から参照できる。
これにより、章番号が変更されても参照が自動的に更新される。

\section{セクションの構成}

各章は複数のセクションで構成される。

\subsection{セクションの階層}

LaTeX では以下の階層構造が使用できる:

\begin{enumerate}
  \item \verb|\chapter{}|: 章 (1, 2, 3, ...)
  \item \verb|\section{}|: 節 (1.1, 1.2, 1.3, ...)
  \item \verb|\subsection{}|: 小節 (1.1.1, 1.1.2, 1.1.3, ...)
  \item \verb|\subsubsection{}|: 小小節 (1.1.1.1, 1.1.1.2, 1.1.1.3, ...)
\end{enumerate}

\subsection{適切な階層の選択}

論文の構成を考える際は、以下の点に注意する:

\begin{itemize}
  \item 階層が深くなりすぎないようにする(通常は subsection まで)
  \item 各セクションには適切な量の内容を含める
  \item セクションの粒度を統一する
\end{itemize}

\section{論文の一般的な構成}

片山徹郎研究室で主に使われてきた、修士論文の一般的な構成は以下の通りである:

\begin{enumerate}
  \item はじめに
  \item 研究の準備(使用ツール、ライブラリの紹介)
  \item 機能
  \item 実装
  \item 適用例
  \item 考察
  \item おわりに(まとめと今後の課題)
\end{enumerate}

この構成が全てではなく、研究内容に合わせて適宜変更されることが多い。
例えば、卒業論文で開発したツールの拡張を行う場合は、拡張した箇所の説明を追加することが多い。
詳細な構成などは、片山研の \verb|graduate| フォルダにある論文を参考にすると良い。

\section{本テンプレートの構成}

本テンプレートでは、以下の構成でサンプルを提供している:

\begin{itemize}
  \item 第\ref{cha:Introduction}章: テンプレートの使い方
  \item 第\ref{cha:Preparation}章: 基本的な LaTeX の書き方
  \item 第\ref{cha:Function}章: 図、表、数式の挿入方法
  \item 第\ref{cha:Implementation}章: ソースコードの挿入方法
  \item 第\ref{cha:Indication}章: 参考文献の引用方法
  \item 第\ref{cha:Evaluation}章: 章立てとセクション構成
  \item 第\ref{cha:Conclusion}章: まとめ
\end{itemize}

実際の論文執筆時は、各章の内容を自分の研究内容に置き換えて使用する。
